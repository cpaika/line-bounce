% ====================================================
% Introduction Chapter
% ====================================================

\chapter{Introduction}

This document is a template for a software requirements specification,
based on Karl Wieger's template (see
\url{http://www.processimpact.com/process_assets/srs_template.doc}). It
is intended to be used as a template for teams in the subjects
SWEN30004 (Software Engineering Project) and SWEN30007 (Software
Systems Project) at the University of Melbourne.

The template is quite detailed, and is probably too heavy weight for
teams in these subjects, so it is unlikely that teams will use all
sections. This is merely a guide as to the types of information that
are found in many SRS documents, and also to serve as an example
\LaTeX\ document.

\section{Purpose}

Identify the product whose software requirements are specified in this
document, including the revision or release number. Describe the scope
of the product that is covered by this SRS, particularly if this SRS
describes only part of the system or a single subsystem.



\section{Document Conventions}

Describe any standards or typographical conventions that were followed
when writing this SRS, such as fonts or highlighting that have special
significance. For example, state whether priorities  for higher-level
requirements are assumed to be inherited by detailed requirements, or
whether every requirement statement is to have its own priority.

\section{Intended Audience and Reading Suggestions}

Describe the different types of reader that the document is intended
for, such as developers, project managers, marketing staff, users,
testers, and documentation writers. Describe what the rest of this SRS
contains and how it is organized. Suggest a sequence for reading the
document, beginning with the overview sections and proceeding through
the sections that are most pertinent to each reader type.


\section{Project Scope}

Provide a short description of the software being specified and its
purpose, including a problem definition, stakeholders, relevant
benefits, objectives, and goals. Relate the software to corporate
goals or business strategies. If a separate vision and scope document
is available, refer to it rather than duplicating its contents
here. An SRS that specifies the next release of an evolving product
should contain its own scope statement as a subset of the long-term
strategic product vision.

\section{References}

List any other documents or Web addresses to which this SRS
refers. These may include user interface style guides, contracts,
standards, system requirements specifications, use case documents, or
a vision and scope document. Provide enough information so that the
reader could access a copy of each reference, including title, author,
version number, date, and source or location.

